\documentclass{article}
\usepackage{hyperref}
\usepackage{graphicx}
\usepackage{float}


\title{Analiza wypadków samochodowych oraz danych pogodowych w Polsce}
\author{Jakub Śliwka}
\date{15 May 2022}

\begin{document}

\maketitle

\section{Wstęp}
Dane zostały pobrane z dzięki uprzjmości serwisów: \url{https://policja.pl/} oraz \url{https://danepubliczne.imgw.pl}.
Polska jest jednym z czołowych krajów w Europie pod względem liczby śmiertelnych wypadków samochodowych w przeliczeniu na milion mieszkańców.
Pod tym względem tylko Rumunia, Bułgaria, Litwa i Chorwacja mają gorsze statystyki.x



\subsection{First Subsection}
\subsubsection{First Subsubsection}

\section{Text Styles}
This text is written in \textit{italic}, this text is \underline{underlined}, this text is \textbf{bold} and this \textbf{\textit{\underline{has everything}}}.

\section{Environments}
\subsection{Lists}
Unordered lists
\begin{itemize}
    \item this
    \item is
    \item a 
    \item list
\end{itemize}
Ordered lists
\begin{enumerate}
    \item this
    \item is
    \item a 
    \item list
\end{enumerate}

\subsection{Equations}
We have inline equations, e.g., the delta equation is $\Delta=b^2-4ac$, but we also have full equations, that we like:
\begin{equation}
    \label{eq_sum}
    \sum_{n=1}^{4} n = 1 + 2 + 3 + 4 = 10 
\end{equation}

Comment to (\ref{eq_sum}) equation
\begin{equation}
    \int_{a}^b f(x)dx = tmp_text
    \label{eq_integral}
\end{equation}

Comment to (\ref{eq_integral}) equation

\section{Tables}

\begin{table}[H]
    \centering
    \begin{tabular}{|l|c|r|}
    \hline
    Column A with Data & Column C with Data & \textbf{Column C with Data} \\ \hline
    data 1             & data 3             & \textbf{data 5}             \\ \hline
    data 2             & data 4             & \textbf{data 6}             \\ \hline
    \end{tabular}
    \caption{Table nr 1}
\end{table}
Example table generator: \url{https://www.tablesgenerator.com/}

\section{Images}
\LaTeX\ allows easy inclusion of images in the document.

\begin{figure}[htbp]
    \centering
    \includegraphics[scale=0.3]{Jupiter_New_Horizons.jpg}
    \caption{Jupiter or something idk [\ref{jupiter}]}
    \label{jupiter}
\end{figure}

\begin{thebibliography}{9}
    \bibitem{jupiterb}
    en.wikipedia.org  \emph{Jupiter New Horizons}
    Accessed: 04 MArch, 2024.
\end{thebibliography}

\end{document}
